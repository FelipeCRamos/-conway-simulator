A programming language project that consists on the implementation of Conway\textquotesingle{}s Game of Life using C++ language.

Even with the word \char`\"{}game\char`\"{} in the name, it\textquotesingle{}s not a iteractive play, instead, it\textquotesingle{}s a simulation which we can observe the behavior of a given {\itshape config} of cells.

\subsection*{Rules of the game}


\begin{DoxyEnumerate}
\item If a cell is alive, but the number of nearby cells is {\bfseries ≤ 1}, on next generation it will {\bfseries die by loneliness}.
\item If a cell is alive and and {\bfseries 4+$\ast$$\ast$ nearby cells are alive, on next generation it will $\ast$$\ast$die suffocated}.
\item If an alive cell has 2 ≥ {\bfseries X} ≥ 3 nearby cells alive, on next generation it will {\bfseries remain alive}.
\item If a cell is current dead and has {\bfseries exacly} {\bfseries 3} nearby cells alive, on next generation it will {\bfseries born}.
\item All the events needs to happen on the exact same time, therefore, cells that are dying can help others to born, but can\textquotesingle{}t prevent the death of others by reducing overpopulation. On the same way, cells that are being born will not help to preserve or kill alive cells on the previous generation.
\end{DoxyEnumerate}

\subsection*{Compile}

First of all, you need to make sure that have these dependencies installed onto your computer\+:


\begin{DoxyItemize}
\item {\ttfamily g++}
\item {\ttfamily git}
\end{DoxyItemize}

If you\textquotesingle{}re on a debian-\/based distro (such as ubuntu), you can use the following command to install both\+:


\begin{DoxyCode}
sudo apt-get install git g++
\end{DoxyCode}


Then, you can clone this repo. For doing it, type the command\+:


\begin{DoxyCode}
git clone https://github.com/FelipeCRamos/conway-simulator.git;
cd conway-simulator
\end{DoxyCode}
 And then you can execute {\ttfamily make} to compile the entire program.

\subsubsection*{Execute}

To execute the program, you can do like\+:


\begin{DoxyCode}
./conway [initial\_config.dat]
\end{DoxyCode}


Knowing that the {\ttfamily initial\+\_\+config.\+dat} is the desired initial config input that the program will recieve.

If you want to save the log\+:


\begin{DoxyCode}
./conway [initial\_config.dat] > [logname.txt]
\end{DoxyCode}


\paragraph*{Initial config parsing}

In order to have a reference file, you need to get some informations like\+:


\begin{DoxyItemize}
\item Alive {\ttfamily char} representation
\item Size of the config
\item The config itself
\end{DoxyItemize}

And write like the example from below\+:


\begin{DoxyCode}
height width
alive\_cell\_char
<config>
\end{DoxyCode}


An example of a 5x5 with {\ttfamily $\ast$} representing the alive cells (anything different will be considered as a dead cell).


\begin{DoxyCode}
5 5
*
.*..*
*..**
***..
...**
*.*.*
\end{DoxyCode}


\paragraph*{Random Generation}

If you\textquotesingle{}re a lazy person, and want to see big canvas screens populated by cells, you can give {\ttfamily random\+\_\+it()} function a chance. Read the {\ttfamily src/main.\+cpp} file to know more.

\subsubsection*{Authorship}

This project was entirely made by Felipe Ramos with M\+IT license for the discipline of programming language I for the computer science course on \href{http://ufrn.br}{\tt U\+F\+RN}.

{\itshape 13/04/2018} 